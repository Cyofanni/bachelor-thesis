%**************************************************************
% file contenente le impostazioni della tesi
%**************************************************************

%**************************************************************
% Frontespizio
%**************************************************************
\newcommand{\myName}{Giovanni Mazzocchin}                                    % autore
\newcommand{\myTitle}{Analisi ed implementazione di un front-end per la lingua italiana all'interno
del motore di sintesi vocale Speect}                    
\newcommand{\myDegree}{Tesi di laurea triennale}                % tipo di tesi
\newcommand{\myUni}{Università degli Studi di Padova}           % università
\newcommand{\myFaculty}{Corso di Laurea in Informatica}         % facoltà
\newcommand{\myDepartment}{Dipartimento di Matematica}          % dipartimento
\newcommand{\myProf}{Paolo Baldan}                                % relatore
\newcommand{\myLocation}{Padova}                                % dove
\newcommand{\myAA}{2015-2016}                                   % anno accademico
\newcommand{\myTime}{Agosto 2016}                                  % quando


%**************************************************************
% Impostazioni di impaginazione
% see: http://wwwcdf.pd.infn.it/AppuntiLinux/a2547.htm
%**************************************************************

\setlength{\parindent}{14pt}   % larghezza rientro della prima riga
\setlength{\parskip}{0pt}   % distanza tra i paragrafi


%**************************************************************
% Impostazioni di biblatex
%**************************************************************
\bibliography{bibliografia} % database di biblatex 

\defbibheading{bibliography}
{
    \cleardoublepage
    \phantomsection 
    \addcontentsline{toc}{chapter}{\bibname}
    \chapter*{\bibname\markboth{\bibname}{\bibname}}
}

\setlength\bibitemsep{1.5\itemsep} % spazio tra entry

\DeclareBibliographyCategory{opere}
\DeclareBibliographyCategory{web}

\addtocategory{opere}{womak:lean-thinking}
\addtocategory{web}{site:agile-manifesto}

\defbibheading{opere}{\section*{Riferimenti bibliografici}}
\defbibheading{web}{\section*{Siti Web consultati}}


%**************************************************************
% Impostazioni di caption
%**************************************************************
\captionsetup{
    tableposition=top,
    figureposition=bottom,
    font=small,
    format=hang,
    labelfont=bf
}

%**************************************************************
% Impostazioni di glossaries
%**************************************************************
% !TEX encoding = UTF-8
% !TEX TS-program = pdflatex
% !TEX root = ../tesi.tex
% !TEX spellcheck = it-IT

%**************************************************************
% Bibliografia
%**************************************************************

\cleardoublepage
\chapter{Glossario}

\paragraph{binutils}
  formalmente \textit{GNU Binutils}, è un insieme di \textit{tool} binari per \textit{GNU}, comprendente, fra
  gli altri, i programmi \texttt{ld} (\textit{GNU linker}) e \texttt{as} (\textit{GNU assembler}).
\label{glo:binu}


\paragraph{consonante approssimante}
\label{glo:consappr}
   classe di suoni che comprende le \textit{semivocali} o \textit{semiconsonanti}, ossia suoni che
   si trovano al confine tra l'articolazione consonantica e quella vocalica.

\paragraph{consonante fricativa}
\label{glo:consfric}
   classe di suoni prodotti mediante un restringimento tra alcuni organi nella cavità orale, che si avvicinano
   senza chiudersi totalmente come avviene per le consonanti occlusive.

\paragraph{consonante laterale}
\label{glo:conslate}
   classe di suoni prodotti mediante una parziale occlusione del canale orale, provocata dalla lingua che
   ne ostruisce la parte centrale lasciando spazio solo ai lati.

\paragraph{consonante nasale}
\label{glo:consnasa}
  classe di suoni caratterizzata da una risonanza che si realizza quando il canale orale viene ostruito, mentre il velo
  palatino rimane abbassato, permettendo il deflusso dell'aria proveniente dai polmoni e dalle fosse nasali.

\paragraph{consonante occlusiva}
\label{glo:consoccl}
  classe di suoni generati mediante il blocco completo del flusso di aria a livello della bocca,
  della faringe e della glottide.
\paragraph{consonante sorda}
\label{glo:conssord}
  classe di suoni articolati senza la vibrazione delle corde vocali.

\paragraph{consonante vibrante}
\label{glo:consvibr}
  classe di suoni prodotti mediante una debole occlusione intermittente del canale orale, la quale
  si interrompe e ripristina velocemente più volte.

\paragraph{hot swapping} 
\label{glo:hots}
 detto anche \textit{hot plugging}, è la capacità di un sistema di sostituire e caricare componenti
 senza la necessità di un suo spegnimento.

\paragraph{OCaml}
\label{glo:ocam}
  ossia \textit{Objective-Caml}, è la principale implementazione del linguaggio \textit{Caml}.

\paragraph{XSL}
\label{glo:xsl}
  sigla di \textit{e\textbf{X}tensible \textbf{S}tylesheet \textbf{L}anguage}, è il linguaggio di descrizione
  dei fogli di stile per i documenti in formato \textit{XML}.


\paragraph{POS tagging}
\label{glo:postag}
  noto anche come \textit{grammatical tagging} o \textit{word-category disambiguation}, è il processo che si occupa
  dell'associazione di una parte del discorso a ciascuna parola in una frase.
 % database di termini
\makeglossaries


%**************************************************************
% Impostazioni di graphicx
%**************************************************************
\graphicspath{{immagini/}} % cartella dove sono riposte le immagini


%**************************************************************
% Impostazioni di hyperref
%**************************************************************
\hypersetup{
    %hyperfootnotes=false,
    %pdfpagelabels,
    %draft,	% = elimina tutti i link (utile per stampe in bianco e nero)
    colorlinks=true,
    linktocpage=true,
    pdfstartpage=1,
    pdfstartview=FitV,
    % decommenta la riga seguente per avere link in nero (per esempio per la stampa in bianco e nero)
    %colorlinks=false, linktocpage=false, pdfborder={0 0 0}, pdfstartpage=1, pdfstartview=FitV,
    breaklinks=true,
    pdfpagemode=UseNone,
    pageanchor=true,
    pdfpagemode=UseOutlines,
    plainpages=false,
    bookmarksnumbered,
    bookmarksopen=true,
    bookmarksopenlevel=1,
    hypertexnames=true,
    pdfhighlight=/O,
    %nesting=true,
    %frenchlinks,
    urlcolor=webbrown,
    linkcolor=RoyalBlue,
    citecolor=webgreen,
    %pagecolor=RoyalBlue,
    %urlcolor=Black, linkcolor=Black, citecolor=Black, %pagecolor=Black,
    pdftitle={\myTitle},
    pdfauthor={\textcopyright\ \myName, \myUni, \myFaculty},
    pdfsubject={},
    pdfkeywords={},
    pdfcreator={pdfLaTeX},
    pdfproducer={LaTeX}
}

%**************************************************************
% Impostazioni di itemize
%**************************************************************
\renewcommand{\labelitemi}{$\ast$}

%\renewcommand{\labelitemi}{$\bullet$}
%\renewcommand{\labelitemii}{$\cdot$}
%\renewcommand{\labelitemiii}{$\diamond$}
%\renewcommand{\labelitemiv}{$\ast$}


%**************************************************************
% Impostazioni di listings
%**************************************************************
\lstset{
    language=[LaTeX]Tex,%C++,
    keywordstyle=\color{RoyalBlue}, %\bfseries,
    basicstyle=\small\ttfamily,
    %identifierstyle=\color{NavyBlue},
    commentstyle=\color{Green}\ttfamily,
    stringstyle=\rmfamily,
    numbers=none, %left,%
    numberstyle=\scriptsize, %\tiny
    stepnumber=5,
    numbersep=8pt,
    showstringspaces=false,
    breaklines=true,
    frameround=ftff,
    frame=single
} 


%**************************************************************
% Impostazioni di xcolor
%**************************************************************
\definecolor{webgreen}{rgb}{0,.5,0}
\definecolor{webbrown}{rgb}{.6,0,0}


%**************************************************************
% Altro
%**************************************************************

\newcommand{\omissis}{[\dots\negthinspace]} % produce [...]

% eccezioni all'algoritmo di sillabazione
\hyphenation
{
    ma-cro-istru-zio-ne
    gi-ral-din
}

\newcommand{\sectionname}{sezione}
\addto\captionsitalian{\renewcommand{\figurename}{figura}
                       \renewcommand{\tablename}{tabella}}

\newcommand{\glsfirstoccur}{\ap{{[g]}}}

\newcommand{\intro}[1]{\emph{\textsf{#1}}}

%**************************************************************
% Environment per ``rischi''
%**************************************************************
\newcounter{riskcounter}                % define a counter
\setcounter{riskcounter}{0}             % set the counter to some initial value

%%%% Parameters
% #1: Title
\newenvironment{risk}[1]{
    \refstepcounter{riskcounter}        % increment counter
    \par \noindent                      % start new paragraph
    \textbf{\arabic{riskcounter}. #1}   % display the title before the 
                                        % content of the environment is displayed 
}{
    \par\medskip
}

\newcommand{\riskname}{Rischio}

\newcommand{\riskdescription}[1]{\textbf{\\Descrizione:} #1.}

\newcommand{\risksolution}[1]{\textbf{\\Soluzione:} #1.}

%**************************************************************
% Environment per ``use case''
%**************************************************************
\newcounter{usecasecounter}             % define a counter
\setcounter{usecasecounter}{0}          % set the counter to some initial value

%%%% Parameters
% #1: ID
% #2: Nome
\newenvironment{usecase}[2]{
    \renewcommand{\theusecasecounter}{\usecasename #1}  % this is where the display of 
                                                        % the counter is overwritten/modified
    \refstepcounter{usecasecounter}             % increment counter
    \vspace{10pt}
    \par \noindent                              % start new paragraph
    {\large \textbf{\usecasename #1: #2}}       % display the title before the 
                                                % content of the environment is displayed 
    \medskip
}{
    \medskip
}

\newcommand{\usecasename}{UC}

\newcommand{\usecaseactors}[1]{\textbf{\\Attori Principali:} #1. \vspace{4pt}}
\newcommand{\usecasepre}[1]{\textbf{\\Precondizioni:} #1. \vspace{4pt}}
\newcommand{\usecasedesc}[1]{\textbf{\\Descrizione:} #1. \vspace{4pt}}
\newcommand{\usecasepost}[1]{\textbf{\\Postcondizioni:} #1. \vspace{4pt}}
\newcommand{\usecasealt}[1]{\textbf{\\Scenario Alternativo:} #1. \vspace{4pt}}

%**************************************************************
% Environment per ``namespace description''
%**************************************************************

\newenvironment{namespacedesc}{
    \vspace{10pt}
    \par \noindent                              % start new paragraph
    \begin{description} 
}{
    \end{description}
    \medskip
}

\newcommand{\classdesc}[2]{\item[\textbf{#1:}] #2}
