% !TEX encoding = UTF-8
% !TEX TS-program = pdflatex
% !TEX root = ../tesi.tex
% !TEX spellcheck = it-IT

%**************************************************************
\chapter{Conclusioni}
\label{cap:conclusioni}
%**************************************************************

%**************************************************************
%\section{Consuntivo finale}

%**************************************************************
\section{Raggiungimento degli obiettivi}
A conclusione dello \textit{stage}, il \textit{front-end} per l'italiano per \textbf{Speect}
è stato sensibilmente migliorato, fornendo al \textit{back-end} un maggior numero di informazioni corrette,
confermando così che \textbf{Speect} può sostituire \textbf{MaryTTS}
per gran parte delle sue funzionalità.
%**************************************************************
\section{Conoscenze acquisite}
L'attività di \textit{stage} è stata particolarmente formativa in quanto
mi ha permesso di entrare nel mondo della sintesi vocale, ambito
che prima di questa esperienza non conoscevo assolutamente. \\
Durante lo \textit{stage}, con il supporto costante del \textit{tutor} aziendale, ho utilizzato
\textit{software} che prima non conoscevo. \\ I software/linguaggi di programmazione utilizzati
sono sempre stati sottoposti ad analisi critica (sempre discutendo con il \textit{tutor}) 
per individuarne pregi e i difetti prima del loro utilizzo.
 
%**************************************************************
\section{Valutazione personale}
Il lavoro svolto è stato giudicato positivamente dall'azienda, costituendo una base solida 
per continuare nello sviluppo di \textbf{Speect}. Il fatto che dopo il mio \textit{stage} sia il
\textit{tutor} sia il nuovo tirocinante abbiano continuato il mio lavoro conferma questo giudizio positivo. \\
Complessivamente è stata un'esperienza utile alla crescita personale, non solo informatica, soprattutto
perché è stata la mia prima esperienza in ambito aziendale. \\
L'unico aspetto negativo è stato il troppo tempo speso nella correzione di errori banali
causati dalla scarsa documentazione, tempo che avrei potuto impiegare nella realizzazione 
di nuove componenti o di nuovi \textit{test}.

