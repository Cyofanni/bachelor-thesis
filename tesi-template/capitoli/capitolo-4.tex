% !TEX encoding = UTF-8
% !TEX TS-program = pdflatex
% !TEX root = ../tesi.tex
% !TEX spellcheck = it-IT

%**************************************************************
\chapter{Analisi dei requisiti}
\label{cap:boh}
%**************************************************************

%\intro{}\\
\section{Forest useful snippet for  mivoq-nlp-scripts}

\begin{forest}
 for tree={
    font=\ttfamily,
    grow'=0,
    child anchor=west,
    parent anchor=south,
    anchor=west,
    calign=first,
    edge path={
      \noexpand\path [draw, \forestoption{edge}]
      (!u.south west) +(7.5pt,0) |- node[fill,inner sep=1.25pt] {} (.child anchor)\forestoption{edge label};
    },
    before typesetting nodes={
      if n=1
        {insert before={[,phantom]}}
        {}
    },
    fit=band,
    before computing xy={l=15pt},
  }
[mivoq-nlp-scripts
  [AUTHORS
    %[text1.1.1]
    %[text1.1.2]
    %[text1.1.3]
  ]
  [README.md
    %[text1.2.1]
    %[text1.2.2]
  ]
  [bin]
  [experiments]
  [lib]
  [test]
  [xsl]
]
\end{forest}

%Nelle tabelle \ref{tab:requisiti-funzionali}, \ref{tab:requisiti-qualitativi} e \ref{tab:requisiti-vincolo} sono riassunti i requisiti e il loro tracciamento con gli% use case delineati in fase di analisi.

\newpage
\section{Forest useful snippet for syllabification plugin}
\begin{forest}
  for tree={
    font=\ttfamily,
    grow'=0,
    child anchor=west,
    parent anchor=south,
    anchor=west,
    calign=first,
    edge path={
      \noexpand\path [draw, \forestoption{edge}]
      (!u.south west) +(7.5pt,0) |- node[fill,inner sep=1.25pt] {} (.child anchor)\forestoption{edge label};
    },
    before typesetting nodes={
      if n=1
        {insert before={[,phantom]}}
        {}
    },
    fit=band,
    before computing xy={l=15pt},
  }
[syllabification
  [AUTHORS
  %  [text1.1.1]
  %  [text1.1.2]
  %  [text1.1.3]
  ]
  [README
    %[text1.2.1]
    %[text1.2.2]
  ]
  [CMakeLists.txt]
  [Doxyfile]
  [cmake
     [sources.cmake]
  ]
  [src
     [ita\_it\_mivoq.h]
     [ita\_it\_mivoq.c]
     [plugin.c]
  ]
]
\end{forest}

\section{Forest useful snippet for rule based features plugin}
\begin{forest}
  for tree={
    font=\ttfamily,
    grow'=0,
    child anchor=west,
    parent anchor=south,
    anchor=west,
    calign=first,
    edge path={
      \noexpand\path [draw, \forestoption{edge}]
      (!u.south west) +(7.5pt,0) |- node[fill,inner sep=1.25pt] {} (.child anchor)\forestoption{edge label};
    },
    before typesetting nodes={
      if n=1
        {insert before={[,phantom]}}
        {}
    },
    fit=band,
    before computing xy={l=15pt},
  }
[rule\_based\_features]
  [AUTHORS
  %  [text1.1.1]
  %  [text1.1.2]
  %  [text1.1.3]
  ]
  [README
    %[text1.2.1]
    %[text1.2.2]
  ]
  [CMakeLists.txt]
  [Doxyfile]
  [cmake
     [sources.cmake]
  ]
  [src
     [rule\_based\_features.h]
     [rule\_based\_features.c]
     [plugin.c]
  ]
]
\end{forest}



%\begin{table}%
%\caption{Tabella del tracciamento dei requisti funzionali}
%\label{tab:requisiti-funzionali}
%\begin{tabularx}{\textwidth}{lXl}
%\hline\hline
%\textbf{Requisito} & \textbf{Descrizione} & \textbf{Use Case}\\
%\hline
%RFN-1     & L'interfaccia permette di configurare il tipo di sonde del test & UC1 \\
%\hline
%\end{tabularx}
%\end{table}%

%\begin{table}%
%\caption{Tabella del tracciamento dei requisiti qualitativi}
%\label{tab:requisiti-qualitativi}
%\begin{tabularx}{\textwidth}{lXl}
%\hline\hline
%\textbf{Requisito} & \textbf{Descrizione} & \textbf{Use Case}\\
%\hline
%RQD-1    & Le prestazioni del simulatore hardware deve garantire la giusta esecuzione dei test e non la generazione di falsi negativi & - \\
%\hline
%\end{tabularx}
%\end{table}%

%\begin{table}%
%\caption{Tabella del tracciamento dei requisiti di vincolo}
%\label{tab:requisiti-vincolo}
%\begin{tabularx}{\textwidth}{lXl}
%\hline\hline
%\textbf{Requisito} & \textbf{Descrizione} & \textbf{Use Case}\\
%\hline
%RVO-1    & La libreria per l'esecuzione dei test automatici deve essere riutilizzabile & - \\
%\hline
%\end{tabularx}
%\end{table}%
