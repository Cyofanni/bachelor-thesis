% !TEX encoding = UTF-8
% !TEX TS-program = pdflatex
% !TEX root = ../tesi.tex
% !TEX spellcheck = it-IT

%**************************************************************
% Bibliografia
%**************************************************************

\cleardoublepage
\chapter{Glossario}

\paragraph{binutils}
  formalmente \textit{GNU Binutils}, è un insieme di \textit{tool} binari per \textit{GNU}, comprendente, fra
  gli altri, i programmi \texttt{ld} (\textit{GNU linker}) e \texttt{as} (\textit{GNU assembler}).
\label{glo:binu}


\paragraph{consonante approssimante}
\label{glo:consappr}
   classe di suoni che comprende le \textit{semivocali} o \textit{semiconsonanti}, ossia suoni che
   si trovano al confine tra l'articolazione consonantica e quella vocalica.

\paragraph{consonante fricativa}
\label{glo:consfric}
   classe di suoni prodotti mediante un restringimento tra alcuni organi nella cavità orale, che si avvicinano
   senza chiudersi totalmente come avviene per le consonanti occlusive.

\paragraph{consonante laterale}
\label{glo:conslate}
   classe di suoni prodotti mediante una parziale occlusione del canale orale, provocata dalla lingua che
   ne ostruisce la parte centrale lasciando spazio solo ai lati.

\paragraph{consonante nasale}
\label{glo:consnasa}
  classe di suoni caratterizzata da una risonanza che si realizza quando il canale orale viene ostruito, mentre il velo
  palatino rimane abbassato, permettendo il deflusso dell'aria proveniente dai polmoni e dalle fosse nasali.

\paragraph{consonante occlusiva}
\label{glo:consoccl}
  classe di suoni generati mediante il blocco completo del flusso di aria a livello della bocca,
  della faringe e della glottide.
\paragraph{consonante sorda}
\label{glo:conssord}
  classe di suoni articolati senza la vibrazione delle corde vocali.

\paragraph{consonante vibrante}
\label{glo:consvibr}
  classe di suoni prodotti mediante una debole occlusione intermittente del canale orale, la quale
  si interrompe e ripristina velocemente più volte.

\paragraph{hot swapping} 
\label{glo:hots}
 detto anche \textit{hot plugging}, è la capacità di un sistema di sostituire e caricare componenti
 senza la necessità di un suo spegnimento.

\paragraph{OCaml}
\label{glo:ocam}
  ossia \textit{Objective-Caml}, è la principale implementazione del linguaggio \textit{Caml}.

\paragraph{XSL}
\label{glo:xsl}
  sigla di \textit{e\textbf{X}tensible \textbf{S}tylesheet \textbf{L}anguage}, è il linguaggio di descrizione
  dei fogli di stile per i documenti in formato \textit{XML}.


\paragraph{POS tagging}
\label{glo:postag}
  noto anche come \textit{grammatical tagging} o \textit{word-category disambiguation}, è il processo che si occupa
  dell'associazione di una parte del discorso a ciascuna parola in una frase.
