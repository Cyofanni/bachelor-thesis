% !TEX encoding = UTF-8
% !TEX TS-program = pdflatex
% !TEX root = ../tesi.tex
% !TEX spellcheck = it-IT

%**************************************************************
\chapter{Introduzione}
\label{cap:introduzione}
%**************************************************************

%Introduzione al contesto applicativo.\\

%\noindent Esempio di utilizzo di un termine nel glossario \\
%\gls{api}. \\

%\noindent Esempio di citazione in linea \\
%\cite{site:agile-manifesto}. \\

%\noindent Esempio di citazione nel pie' di pagina \\
%citazione\footcite{womak:lean-thinking} \\

%**************************************************************
\section{L'azienda e lo stage}

\href{https://www.mivoq.it/}{Mivoq S.r.L.} è un'azienda operante nel settore delle tecnologie vocali, nata nel 2013 come
\textit{spinoff} dell'\href{http://www.istc.cnr.it/it}{Istituto di Scienze e Tecnologie della Cognizione} del 
\href{https://www.cnr.it/}{Consiglio Nazionale delle Ricerche}. L'azienda propone l'utilizzo delle tecnologie vocali per permettere a chiunque di creare e personalizzare la propria voce sintetica in breve tempo. \\

%**************************************************************
Lo \textit{stage} ha avuto una durata di due mesi per un totale di 320 ore, da maggio 2016 a luglio 2016.
La politica aziendale non adotta orari fissi per i propri dipendenti, tuttavia ho sempre cercato di lavorare
7-8 ore al giorno dal lunedì al venerdì per non superare i due mesi di tirocinio. \\ Ho avuto la possibilità di
lavorare  sia nelle due sedi di Mivoq (Padova e Bassano del Grappa) sia a casa, ma in costante contatto
con il \textit{tutor} aziendale Giulio Paci. \\ Sia io sia il \textit{tutor} abbiamo mantenuto costantemente aggiornati dei \textit{report}
contenenti i \textit{task} assegnati e la descrizione del progresso nel loro svolgimento. 

\begin{figure}[!h] 
    		\centering 
    		\includegraphics[width=0.5\columnwidth]{mivoq-logo} 
   	 	\caption{Logo dell'azienda}
\end{figure}

\section{Descrizione generale di Speect e idea complessiva del progetto}
\textbf{Speect} è una piattaforma di \textit{Text to Speech (TTS)} multilingua sviluppata dal gruppo 
\href{http://www.meraka.org.za/humanLanguage.htm}{Human Language Technologies}
del \href{http://www.csir.co.za/meraka/}{Meraka Institute}, in Sudafrica. \\ Le componenti base del sistema sono implementate in linguaggio \textit{C}, rispettando
rigidamente le direttive dello \textit{standard} \href{http://www.dii.uchile.cl/~daespino/files/Iso_C_1999_definition.pdf}{ISO/IEC 9899:1990}; 
\textbf{Speect} inoltre implementa il paradigma ad oggetti ed è \textit{multithreading}. \\ Il sistema è stato progettato in modo da essere facilmente estensibile
tramite \textit{plug-in}: in particolare, tutti i processi di sintesi (dal \textit{Tokenizer} al \textit{Vocoder}, passando per il \textit{Phonemizer}) 
sono implementati come \textit{plug-in}. \\ L'\textit{engine} di \textbf{Speect}
si occupa invece della definizione dei tipi astratti e della comunicazione tra i vari \textit{plug-in}; esso implementa
inoltre la struttura dati principale di \textbf{Speect}, l'\textit{Heterogeneous Relation Graph}, struttura continuamente
modificata ed estesa dai \textit{plug-in} in esecuzione. \\
Il \textit{plug-in} che si occupa del \hyperref[glo:postag]{\emph{Pos tagging}\glsfirstoccur} all'interno di \textbf{Speect}
è un \textit{wrapper} di \href{https://github.com/mivoq/hunpos}{Hunpos} (al quale hanno lavorato i due tirocinanti
precedenti), un'implementazione \textit{open source} di un algoritmo
che utilizza il \href{https://en.wikipedia.org/wiki/Hidden_Markov_model}{Modello di Markov Nascosto}; \textit{Hunpos} è una 
reimplementazione in linguaggio \hyperref[glo:ocam]{\emph{OCaml}\glsfirstoccur} di 
\href{http://www.coli.uni-saarland.de/~thorsten/tnt/}{TnT}, un \textit{part-of-speech tagger} sviluppato da Thorsten Brants. \\

Il progetto sviluppato in questo \textit{stage} si inquadra in un'attività di medio/lungo termine che consiste nella migrazione 
del motore di sintesi utilizzato dall'azienda da una piattaforma \textit{Java} (\textbf{MaryTTS}) ad una \textit{C} (\textbf{Speect}). \\
\textbf{MaryTTS} è un sistema scritto in \textit{Java}, fornisce dei buoni strumenti per l'analisi interna, tuttavia presenta i seguenti difetti:
   \begin{itemize}
        \item basse \textit{performance} dovute al linguaggio; 
        \item molto codice duplicato e troppe dipendenze; 
        \item struttura ad albero rigida per contenere le informazioni ricavate dal \textit{front-end}: ciò rende complicato
              realizzare alcune analisi linguistiche;
        \item difficoltà nel modificare il \textit{vocoder} mantenendo la compatibilità all'indietro;
        \item carenza di librerie adeguate per \textit{Java} per effettuare il processamento audio;
        \item struttura modulare rigida per le lingue e le voci;
        \item \textit{iOS} non supporta \textit{Java}, e in futuro potrebbe essere interesse dell'azienda sviluppare per questa piattaforma.
   \end{itemize}
L'azienda prevede dunque di poter estendere \textbf{Speect} fino al momento in cui esso conterrà un sottoinsieme delle funzionalità
di \textbf{MaryTTS}, con lo scopo di utilizzarlo come sistema di \textit{TTS} nei servizi e nei prodotti forniti da Mivoq. \\
L'attività primaria dello \textit{stage} è stata dunque l'implementazione di alcuni passaggi di tale migrazione, tra cui il miglioramento
di una voce sintetica in lingua italiana tramite l'estensione e la creazione di alcuni \textit{plug-in}. \\
I due tirocinanti precedenti avevano sviluppato le componenti \textit{Tokenizer}, \textit{POS tagger} e \textit{trascrittore fonetico};
inoltre essi avevano lavorato ad una \textit{suite} di \textit{tool} (principalmente \textit{script bash} e fogli \hyperref[glo:xsl]{\emph{XSL}\glsfirstoccur}) 
per agevolare il confronto tra i due sistemi \textbf{MaryTTS} e \textbf{Speect}, \textit{suite} alla quale ho lavorato anche io per un
 breve periodo (circa tre giorni) all'inizio dello \textit{stage} come introduzione al sistema. \\
Prima del mio \textit{stage} la qualità della voce sintetica italiana era piuttosto bassa, principalmente a causa
dei seguenti motivi:
 \begin{itemize}
   \item il testo in lingua italiana veniva sillabificato secondo le regole della lingua inglese
         in quanto mancava il \textit{plug-in} specifico per l'italiano adatto a questo scopo;
   \item molte \textit{feature} utili alla produzione dell'audio da parte del \textit{vocoder} non venivano calcolate
         o raccolte propriamente, fornendo così al \textit{back-end} un numero insufficiente di informazioni sull'\textit{input}.
 \end{itemize}
Per quanto riguarda il metodo adottato nello sviluppo del progetto, l'assenza di documentazione (se non nei commenti al codice) 
riguardante i \textit{plug-in} sviluppati dai tirocinanti precedenti (sui quali ho lavorato) ha reso necessari uno studio continuo del 
codice sorgente e l'esecuzione di numerosi \textit{test} per comprendere il funzionamento del sistema. \\ 
Ad ogni modo, in alcuni casi, l'interazione con il \textit{tutor} aziendale è stata
indispensabile e fondamentale per la comprensione dei dettagli più ostici.


%*************************************
\section{Problematiche riscontrate}
 \subsection{Documentazione di Speect}
  La documentazione di \textbf{Speect} è piuttosto scarsa e contiene alcuni errori che possono confondere il principiante, inoltre
  gli sviluppatori originali hanno abbandonato il progetto nel 2013, pertanto non esistono aggiornamenti alla documentazione da molto tempo.
  Inoltre, effettuando delle ricerche tramite i principali \textit{search engine}, sembra evidente come \textbf{Speect} non abbia ricevuto molta attenzione
  al di fuori dell'istituto nel quale è stato sviluppato (oltre a Mivoq ovviamente). \\ Pertanto non è stato possibile nemmeno 
  rivolgersi a \textit{forum} o fonti di terze parti per la risoluzione di problemi anche banali (la tipica ricerca su 
  \textit{Google} non restituiva risultati specifici). Per questo motivo la risoluzione di problemi anche banali
  ha richiesto un numero di ore non irrilevante. \\
  È inoltre notevole il fatto che la mancanza/scarsa qualità di documentazione a volte abbia reso necessaria l'analisi del codice sorgente 
  interno all'\textit{engine} di \textbf{Speect}. \\
  Visti i problemi citati sopra, nel corso dello \textit{stage} è stato ritenuto opportuno stilare una lista 
  (condivisa con il \textit{tutor} aziendale) degli errori individuati nella documentazione.
  
%*************************************
\section{Prodotto ottenuto}
Tutto il lavoro svolto durante lo \textit{stage} è rientrato nel percorso tracciato dai due tirocinanti
precedenti, i quali hanno sviluppato parte di un \textit{front-end} minimale per la lingua italiana per \textbf{Speect}.
Il mio compito è stato dunque lo sviluppo di nuove funzionalità per tale \textit{front-end} allo scopo di 
renderlo più maturo, ottenendo una voce in lingua italiana di qualità superiore.\\
In particolare, le (macro)attività svolte durante il tirocinio sono state le seguenti:
    \begin{itemize} 
       \item miglioramento del \textit{plug-in} responsabile della scrittura in formato \textit{MaryXML}
             dei dati interni all'\textit{HRG} di \textbf{Speect}; sono inoltre stati estesi i \textit{test} esistenti
             per verificare il comportamento del \textit{plug-in} dopo le estensioni;
       \item sviluppo di un sillabificatore completo per l'italiano e dei \textit{test} per 
             verificarne il corretto funzionamento su un grande numero di possibili \textit{input};
       \item analisi ed implementazione di alcune componenti per il calcolo 
             delle \textit{feature} utili per il \textit{vocoder} (è stato cioè migliorato il passaggio
             dei dati dal \textit{front-end} al \textit{back-end}). \\
    \end{itemize}
Dopo il termine del mio \textit{stage} l'azienda procederà con lo sviluppo di \textbf{Speect}, avendo come obiettivo
un prodotto dotato di tutte le funzionalità di \textbf{MaryTTS} e in grado di sostituirlo per l'utilizzo aziendale. 


%**************************************************************
\newpage
\section{Organizzazione del testo}

\begin{description}
    \item[{\hyperref[cap:analisi-requisiti]{Il secondo capitolo}}] presenta brevemente il sistema e descrive i requisiti del progetto;
    
    \item[{\hyperref[cap:speect]{Il terzo capitolo}}] descrive in profondità le componenti fondamentali del \textit{software} \textbf{Speect};
    
   % \item[{\hyperref[cap:analisi-requisiti]{Il quarto capitolo}}] approfondisce ...
    
    \item[{\hyperref[cap:progettazione-realizzazione]{Il quarto capitolo}}] descrive le scelte progettuali e di codifica, dal sillabificatore
      per l'italiano ai \textit{plug-in} per il calcolo delle \textit{feature};
    
    \item[{\hyperref[cap:verifica-validazione]{Il quinto capitolo}}] descrive le attività di verifica e validazione svolte sulle
      componenti prodotte durante lo \textit{stage};
    
    \item[{\hyperref[cap:conclusioni]{Il sesto capitolo}}] contiene alcune considerazioni finali sullo \textit{stage}.
\end{description}

Riguardo la stesura del testo, relativamente al documento sono state adottate le seguenti convenzioni tipografiche:
\begin{itemize}
	\item gli acronimi, le abbreviazioni e i termini ambigui o di uso non comune menzionati vengono definiti nel glossario, situato alla fine del presente documento;
	\item per le occorrenze dei termini riportati nel glossario viene utilizzata la seguente nomenclatura: \emph{parola}\glsfirstoccur;
	\item i termini in lingua straniera o facenti parti del gergo tecnico sono evidenziati con il carattere \emph{corsivo}.
\end{itemize}
